%%%% ijcai19.tex

\typeout{IJCAI-19 Instructions for Authors}

% These are the instructions for authors for IJCAI-19.

\documentclass{article}
\pdfpagewidth=8.5in
\pdfpageheight=11in
% The file ijcai19.sty is NOT the same than previous years'
\usepackage{ijcai19}

% Use the postscript times font!
\usepackage{times}
\usepackage{soul}
\usepackage{url}
\usepackage[hidelinks]{hyperref}
\usepackage[utf8]{inputenc}
\usepackage[small]{caption}
\usepackage{graphicx}
\usepackage{amsmath}
\usepackage{booktabs}
\usepackage{algorithm}
\usepackage{algorithmic}
% \usepackage{epstopdf}
\usepackage{makecell}
\usepackage{float}
\urlstyle{same}

% the following package is optional:
%\usepackage{latexsym}

% Following comment is from ijcai97-submit.tex:
% The preparation of these files was supported by Schlumberger Palo Alto
% Research, AT\&T Bell Laboratories, and Morgan Kaufmann Publishers.
% Shirley Jowell, of Morgan Kaufmann Publishers, and Peter F.
% Patel-Schneider, of AT\&T Bell Laboratories collaborated on their
% preparation.

% These instructions can be modified and used in other conferences as long
% as credit to the authors and supporting agencies is retained, this notice
% is not changed, and further modification or reuse is not restricted.
% Neither Shirley Jowell nor Peter F. Patel-Schneider can be listed as
% contacts for providing assistance without their prior permission.

% To use for other conferences, change references to files and the
% conference appropriate and use other authors, contacts, publishers, and
% organizations.
% Also change the deadline and address for returning papers and the length and
% page charge instructions.
% Put where the files are available in the appropriate places.

\title{Watercolor rendering and caustics effect for underwater scene} % TODO: TRANSLATE TO FRENCH MAYBE

% Multiple author syntax (remove the single-author syntax above and the \iffalse ... \fi here)
% Check the ijcai19-multiauthor.tex file for detailed instructions

\author{
Arnaud Pare Vogt
\and
Mehdi Chaid
\affiliations
Département GIGL Polytechnique Montreal\\
\emails
arnaud.parevogt@gmail.com, % TODO: UPDATE WITH POLYMTL EMAIL
mehdi.chaid@polymtl.ca
}

\begin{document}

\maketitle

\begin{abstract}
TODO: Short summary about the project here.
\end{abstract}

\section{Introduction}

TODO: Talk about inspirations here. \medskip \par 

\noindent
TODO: Talk about the watercolor and caustics papers here.

% EXAMPLE HOW TO CITE HERE
% The first proof that shapes are fundamental in computer vision came from a very influential study on cognition, 
% by \cite{hubel1959receptive}, who described how biological neurons could extract features from images,
% amongst which certain types of neurons were activated specifically by edges. \medskip \par 

\newpage
\section{Scene}

\subsection{Framework}

TODO: Talk about learnopengl.com and the base framework here.

\subsection{Content}

TODO: Talk about the scene content here.

\subsection{Pipeline}

TODO: Talk about the shader pass pipeline here.

\newpage
\section{Watercolor rendering}

TODO: Talk about the watercolor rendering pipeline here.

\subsection{Cel shading}

TODO: Talk about cel shading here.

\subsection{Normal smoothing}

TODO: Talk about normal smoothing here.

\subsection{Morphological smoothing}

TODO: Talk about morphological smoothing here.

\subsection{Paper effects}

TODO: Talk about paper effects here.

\subsection{Edge darkening}

TODO: Talk about edge darkening here.

\subsection{Pigment density variation}

TODO: Talk about pigment density variation here.

\newpage
\section{Caustics}

TODO: Talk about caustics here.

\section{Underwater effects}

TODO: Talk about underwater effects here.

\section{Results}

TODO: Showcase more results here.

% \newpage
\section{Future Work}

TODO: Talk about possible improvements here.

\section{Conclusion}

TODO: Conclude here.

\newpage
\appendix

%% The file named.bst is a bibliography style file for BibTeX 0.99c
\bibliographystyle{named}
\bibliography{ijcai19}

% The reference section will autofill when we add \cite{} in the report.

\end{document}

